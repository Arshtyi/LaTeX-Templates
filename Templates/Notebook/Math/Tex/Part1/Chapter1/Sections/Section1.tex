\section{线性映射}
\subsection{映射}
\dfn{映射}{映射}{ 对于集合$A$、$B$,定义映射
    \begin{align*}
        \bm{f}:
        A         & \longrightarrow B \\
        \forall a & \longmapsto
        \exists b =\bm{f}\left(a\right)
    \end{align*}
    且
    \[
        \mathrm{Im}\, \bm{f} =\bm{f}
        \left(A\right)
        =\left\{\bm{f}\left(a\right)\mid
        a \in A\right\}
    \]
    称为$\bm{f}$的像集. 特别地,
    \[
        \bm{f}:A \longrightarrow A
    \]
    称为一个变换.
}
\dfn{单射、满射、双射}{单射、满射、双射}{
    \[
        \text{双射/一一对应}
        \begin{cases*}
            \text{满射(\textup{onto}):}
            \mathrm{Im}\,\bm{f} = B \\
            \text{单射(\textup{1-1}):}a_1
            \neq a_2 \Longleftrightarrow
            \bm{f}\left(a_1\right) \neq
            \bm{f}\left(a_2\right)
        \end{cases*}
    \]
}
\exa{}{}{设
    \begin{align*}
        det:M_n\left(\mathbb{R}\right) & \longrightarrow \mathbb{R}      \\
        \bm{A}                         & \longmapsto \left|\bm{A}\right|
    \end{align*}
    是满射,$n\geqslant 2$是不是单射.
}
\dfn{恒等映射}{恒等映射}{
    如果一个变换是一个将任一元素变换为自身的
    双射,称为一个恒等映射,记作$\bm{1}_A$或
    $\bm{I}_A$.}
\rem{}{}{ 恒等映射是特别的变换,
    后者只是变换到同一集合,前者在双射的基础上
    还要强调
    任一元素均变换到自身. 这比变换的要求严格得多.

    事实上,在变换的基础上,如果任一元素均映射到自身,
    就可以确定这是双射了. 因此上面的定义其实是
    有重合的.}
\thm{}{映射相同}{
    设两个映射
    \[
        \bm{f},\bm{g}:A\longrightarrow
        B
    \]
    则$\bm{f}=\bm{g}$
    当且仅当$\forall a \in A,\bm{f}
        \left(a
        \right) = \bm{g}
        \left(a\right)$
}
\dfn{映射复合}{映射复合}{设
    \[
        \bm{f}
        :A \longrightarrow B ,
        \bm{g}
        :B \longrightarrow C
    \]
    则有复合映射
    \[
        \bm{g} \circ \bm{f} :
        A \longrightarrow
        C
    \]
    \[
        \bm{g} \circ \bm{f}
        \left(a\right)
        = \bm{g}
        \left(\bm{f}
        \left(a\right)\right)
    \]
    若$\bm{h}:
        C \longrightarrow D$,有结合律:
    \[
        \bm{h}
        \circ \left(\bm{g}
        \circ \bm{f}
        \right)
        = \left(\bm{h}
        \circ \bm{g}
        \right) \circ \bm{f}
    \]}
\dfn{逆映射}{逆映射}{
    若$\bm{f}:A\longrightarrow B$、
    $\bm{g} : B \longrightarrow A$
    使得
    \[
        \begin{cases*}
            \bm{g}\circ\bm{f}=\bm{1}_A \\
            \bm{f}\circ\bm{g}=\bm{1}_B
        \end{cases*}
    \]
    那么称$\bm{g}$为$\bm{f}$的逆映射,
    记作$\bm{g} = \bm{f}^{-1}$.
}
\lem{逆映射的存在性}{逆映射的存在性}{
    $\bm{f} : A \longrightarrow B$的
    逆映射存在当且仅当$\bm{f}$为双射.
    \begin{proof}
        先证充分性,设双射$\bm{f}$,$\forall b\in B$,由满性知
        $\exists a\in A$使得$b=\bm{f}\left(a\right)$,由单性知
        这样的$a$唯一. 构造
        \begin{align*}
            \bm{g}:B  & \longrightarrow A \\
            \forall b & \longmapsto a
        \end{align*}
        此处$a$是唯一满足$\bm{f}\left(a\right)=b$的$a$.
        这是一个映射且
        \[
            \begin{cases*}
                \bm{g}\circ\bm{f}=\bm{1}_A \\
                \bm{f}\circ\bm{g}=\bm{1}_B
            \end{cases*}
        \]
        于是这是符合要求的逆映射.

        再证必要性,即有$\bm{g}=\bm{f}^{-1}$. 先证明单性,设
        $a_1,a_2\in A$使得$\bm{f}\left(a_1\right)=
            \bm{f}\left(a_2\right)$,两边复合$\bm{g}$即得
        $a_1=a_2$.

        考虑满性,$\forall b\in B$,因为
        \[
            \bm{f}\left(\bm{g}\left(b\right)\right)=b
        \]
        故$\bm{g}\left(b\right)$是$b$的原像.
    \end{proof}
}
\subsection{线性映射}
\dfn{线性映射}{线性映射}{
    设$\mathbb{K}$上的线性空间$V$、
    $U$,映射\[
        \bm{\varphi }:V \longrightarrow
        U
    \]

    若$\forall \bm{\alpha}$、$\bm{\beta} \in V$,$\forall k \in \mathbb{K}$
    \begin{enumerate}[label=\textup{(\arabic*)}]
        \item $\bm{\varphi}\left(
                  \bm{\alpha}+\bm{\beta}
                  \right)=\bm{\varphi}\left(
                  \bm{\alpha}
                  \right)   +
                  \bm{\varphi}\left(
                  \bm{\beta}
                  \right)$
        \item $\bm{\varphi}\left(k\bm{\alpha}
                  \right)=k\bm{\varphi}\left(
                  \bm{\alpha}
                  \right)$
    \end{enumerate}
    则称为一个线性映射.

    特别地,$\bm{\varphi}:V
        \longrightarrow V$
    称为线性变换.

    当$\bm{\varphi}$作为映射是单射或满射时,
    称为是单或满线性映射.

    当$
        \bm{\varphi}$
    作为映射是双射时,称为(线性)同构
    (与线性空间之间的同构的定义相同),
    记作$V\cong U$.

    $V=U$时$V$自身上的同构称为自同构.
}
\rem{}{}{定义线性空间之间的同构的时候,我们先要求双射,再要求
    保持线性组合,\cref{def:线性映射}反过来先要求线性组合再要求双射.
    两种定义是相同的.}
\exa{}{}{
    \[
        \bm{\varphi}:
        \forall \bm{\alpha}\in V
        \longmapsto \bm{0}_U
    \]
    这是一个零线性映射$\bm{0}:V\longrightarrow U$.
}
\exa{}{}{
    线性空间$V$上的线性变换
    \begin{align*}
        \bm{1}_V:
        V           & \longrightarrow V \\
        \bm{\alpha} & \longmapsto
        \bm{\alpha}
    \end{align*}
    称为一个恒等映射或者恒等变换,
    也是一个自同构.
    记作$\bm{1}_V$或
    $\bm{I}_V$或
    $\mathbf{Id}_V$或
    $\bm{I}$.
}
\exa{}{}{
    需要说明的是,变换并不等于恒等映射,
    变换只强调$V \longrightarrow V$,恒等变换
    还需要$\bm{\alpha} \longmapsto
        \bm{\alpha}$,
    恒等映射是特别的变换.
}\exa{}{2}{
    设线性空间$\displaystyle
        V = \mathbb{K}^n$,$\displaystyle
        U = \mathbb{K}^m$,$\bm{A}
        \in M_{m \times n}\left(
        \mathbb{K}
        \right)$,定义
    \begin{align*}
        \bm{\varphi}_{\bm{A}}
        : \mathbb{K}^n &
        \longrightarrow \mathbb{K}^m \\
        \bm{\alpha}    & \longmapsto
        \bm{A\alpha}
    \end{align*}
    是一个线性映射.
}
\rem{}{}{
    可以看出,线性映射的几何意义就是矩阵乘法.
    \cref{ex:2}非常重要.
}
\exa{}{}{
    \begin{align*}
        \bm{\varphi}:
        \mathbb{K}_n &
        \longrightarrow
        \mathbb{K}^n   \\
        \left(a_1,a_2,\cdots
        ,a_n\right)  &
        \longmapsto
        \left(
        a_1,a_2,\cdots,a_n
        \right)'
    \end{align*}
    是线性同构.
}
\exa{}{}{
    设$V_{\mathbb{K}}$的一组基为
    $\left\{
        \bm{e}_1,\bm{e}_2,\cdots,\bm{e}_n
        \right\}$,则
    \begin{align*}
        \bm{\varphi}:
        V & \longrightarrow \mathbb{K}^n \\
        \bm{\alpha} =
        \sum_{i = 1}^{n}a_i\bm{e}_i
          & \longmapsto\left(
        a_1,a_2,\cdots,a_n
        \right)'
    \end{align*}
    是一个线性同构.
}
\exa{}{}{
    设$\mathbb{K}$上的线性空间$V$,
    取定$k \in \mathbb{K}$,
    则\begin{align*}
        \bm{\varphi}:
        V           & \longrightarrow V \\
        \bm{\alpha} & \longmapsto
        k\bm{\alpha}
    \end{align*}
    这样一个线性变换称为数量/数乘变换.
}
\exa{}{}{
    设$\left[0,1\right]$上的实无穷次可微函数
    全体组成的实线性空间为$\displaystyle
        C^{\infty}\left[0,1\right]$,则有
    求导变换
    \begin{align*}
        \bm{\varphi}:
        C^{\infty}\left[0,1\right] &
        \longrightarrow
        C^{\infty}\left[0,1\right]               \\
        f\left(x\right)
                                   & \longmapsto
        \frac{\mathrm{d}}{\mathrm{d}x}
        f\left(x\right) = f'\left(x\right)
    \end{align*}
    是$\displaystyle
        C^{\infty}\left[0,1\right]$
    上的线性变换.
}
\pro{线性映射的若干性质}{线性映射的若干性质}{
    设线性映射$\bm{\varphi}:
        V \longrightarrow U$,则
    \begin{enumerate}[label=\textup{(\arabic*)}]
        \item $\bm{\varphi}\left(\bm{0}_V
                  \right)=\bm{0}_U$
        \item $\forall \bm{\alpha},\bm{\beta}
                  \in V,k,l \in \mathbb{K},$
              则
              $\bm{\varphi}\left(
                  k\bm{\alpha} + l\bm{\beta}
                  \right)=k\bm{\varphi}\left(
                  \bm{\alpha}
                  \right)+l\bm{\varphi}\left(
                  \bm{\beta}
                  \right)$即保持线性组合
        \item 若$\bm{\varphi}$为同构,
              则$\bm{\varphi}^{-1}$
              也是同构
        \item 若$\bm{\varphi}:V\longrightarrow U$
              和$\bm{\psi}:U \longrightarrow
                  W$均为线性映射或同构,则$\bm{\psi}
                  \circ \bm{\varphi}:
                  V \longrightarrow W$
              是一个线性
              映射或同构
    \end{enumerate}
    \begin{proof}
        \begin{enumerate}[label=\textup{(\arabic*)}]
            \item \[
                      \bm{\varphi}\left(\bm{0}_V\right)=
                      \bm{\varphi}\left(\bm{0}_V+\bm{0}_V\right)
                      =\bm{\varphi}\left(\bm{0}_V\right)+\bm{\varphi}\left(\bm{0}_V\right)
                      =\bm{0}_U
                  \]
            \item 拆分即可.
            \item 因为$\bm{\varphi}$为双射,则$\bm{\varphi}^{-1}$为双射,只要证
                  $\bm{\varphi}^{-1}$保持线性组合即可.
                  任取$\bm{x},\bm{y}\in U,k,l\in\mathbb{K}$,
                  只需证
                  \[
                      \bm{\varphi}^{-1}\left(k\bm{x}+l\bm{y}\right)=
                      k\bm{\varphi}^{-1}\left(\bm{x}\right)+l
                      \bm{\varphi}^{-1}\left(
                      \bm{y}\right)
                  \]
                  作
                  \begin{align*}
                      \bm{\varphi}\left(
                      \bm{\varphi}^{-1}\left(
                          k\bm{x}+l\bm{y}\right)-
                      k\bm{\varphi}^{-1}\left(\bm{x}\right)+l
                      \bm{\varphi}^{-1}\left(
                          \bm{y}\right)
                      \right)=k\bm{x}+l\bm{y}-k\bm{x}-l\bm{y}=\bm{0}_U=\bm{\varphi}\left(\bm{0}_
                      V\right)
                  \end{align*}
                  于是由单性即得.

            \item 首先$\bm{\psi}\circ \bm{\varphi}$显然为双射,
                  那么
                  \begin{align*}
                      \bm{\psi}\circ \bm{\varphi}\left(k\bm{\alpha}+l\bm{\beta}\right) & =\bm{\psi}\left(k\bm{\varphi}\left(\bm{\alpha}\right)+l\bm{\varphi}\left(\bm{\beta}\right)\right)       \\
                                                                                       & =k\bm{\psi}\circ\bm{\varphi}\left(\bm{\alpha}\right)+l\bm{\psi}\circ\bm{\varphi}\left(\bm{\beta}\right)
                      \qedhere\end{align*}
        \end{enumerate}
    \end{proof}
}
\cor{线性同构的存在性}{线性同构的存在性}{
    事实上,应该说两个线性空间存在线性同构.
    \begin{enumerate}[label=\textup{(\arabic*)}]
        \item \label{1:1}线性同构是一种等价关系
        \item \label{1:2}两个线性空间存在同构当且仅当
              它们的维数相同
    \end{enumerate}
    \begin{proof}
        证明\ref{1:2},先证必要性,设线性同构$\bm{\varphi}:
            V\longrightarrow U$,任取$V$的基$\left\{\bm{e}_1,
            \bm{e}_2,\cdots,\bm{e}_n\right\}$,从而
        $\left\{\bm{\varphi}\left(\bm{e}_1\right),
            \bm{\varphi}\left(\bm{e}_2\right),\cdots,\bm{\varphi}\left(\bm{e}_n\right)\right\}$
        线性无关.

        任取$\bm{x}\in U,\exists \bm{\alpha}\in V$,s.t.
        $\bm{\varphi}\left(\bm{\alpha}\right)=\bm{x}$.
        设$\bm{\alpha}=a_1\bm{e}_1+a_2\bm{e}_2+\cdots
            +a_n\bm{e}_n$,那么
        \[
            \bm{x}=a_1\bm{\varphi}\left(\bm{e}_1\right)+
            a_2\bm{\varphi}\left(\bm{e}_2\right)+\cdots+
            a_n\bm{\varphi}\left(\bm{e}_n\right)
        \]
        于是$\left\{\bm{\varphi}\left(\bm{e}_1\right),
            \bm{\varphi}\left(\bm{e}_2\right),\cdots,\bm{\varphi}\left(\bm{e}_n\right)\right\}$
        是$U$的一组基. 即$\dim\,V=\dim\,U=n$.

        然后是充分性,设$\dim\,V=\dim\,U=n$,取$V$的一组基
        $\left\{\bm{e}_1,\bm{e}_2,\cdots,\bm{e}_n\right\}$和
        $U$的一组基$\left\{\bm{f}_1,\bm{f}_2,\cdots,\bm{f}_n\right\}$.
        构造线性同构:
        \begin{align*}
            \bm{\varphi}_V:
            V & \longrightarrow \mathbb{K}^n \\
            \bm{\alpha} =
            \sum_{i = 1}^{n}a_i\bm{e}_i
              & \longmapsto \begin{pmatrix}
                                a_1    \\
                                a_2    \\
                                \vdots \\
                                a_n
                            \end{pmatrix}
        \end{align*}
        和
        \begin{align*}
            \bm{\varphi}_U:
            U & \longrightarrow \mathbb{K}^n \\
            \bm{x} =
            \sum_{i = 1}^{n}b_i\bm{f}_i
              & \longmapsto \begin{pmatrix}
                                b_1    \\
                                b_2    \\
                                \vdots \\
                                b_n
                            \end{pmatrix}
        \end{align*}
        考虑$\bm{\varphi}_U^{-1}\circ \bm{\varphi}_V$即是线性同构.
    \end{proof}
}
\rem{}{}{
    两个线性空间存在同构时,将存在无数个线性同构,
    但这并不意味着这两个线性空间之间的
    任一线性映射均是线性同构. 即存在无穷多个并不等于
    任取一个均满足.
}
\exa{}{}{
    考虑$\bm{I}:V\longrightarrow V$为线性同构,但是零映射
    $\bm{0}:V\longrightarrow V$不仅不是线性同构,甚至
    都不是单射.
}
\rem{}{}{
    不同数域上的线性空间之间的映射不是线性映射.
}