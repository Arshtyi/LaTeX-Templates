\newpage
\section{不变子空间}
\subsection{定义}
\dfn{不变子空间}{不变子空间}{
    设线性映射$\bm{\varphi}\in \mathcal{L}\left(V_{\mathbb{K}}\right)$,
    $U$是$V$的子空间. 若
    \[
        \bm{\varphi}\left(U\right)\subseteq U
    \]
    则称$U$为线性变换
    $\bm{\varphi}
    $的不变子空间,或称为$\bm
        {\varphi}$-不变子空间.
}
\dfn{}{线性变换的限制}{
    已知$U$为$\bm{\varphi}$-不变子空间,
    若进一步将$\bm{\varphi}$的定义域限制在$U$上,则这个$\bm{\varphi}$
    称为$U$上的一个线性变换,称为$\bm{\varphi}$诱导出的
    线性变换或者是$\bm{\varphi}$在$U$上的限制,记作
    \[
        \bm{\varphi}\Big |_U:U\longrightarrow U
    \]
}
\exa{}{}{
    设$\bm{\varphi}\in \mathcal{L}\left(V
        \right)$,则一定有两个平凡的$\bm{\varphi}$-不变子空间
    即零子空间$0$和全子空间$V$. 此外,核空
    间$\mathrm{Ker}\,\bm{\varphi}$和像空间$\mathrm{Im}\,\bm{\varphi}$
    存在时也一定是$\bm{\varphi}$-不变子空间.
}
\clm{}{}{
    在此定义的线性变换在$V$不是$0$时一定是非零映射.
}
\rem{}{}{
    事实上,像空间和核空间都是必定存在的,但是当映射是恒等变换时,
    像空间即全子空间,零空间即零子空间. 因此上面应该说是像空间和核空间非平凡或者是
    非平凡的像空间和核空间存在时.
}
\rem{}{}{
    零子空间$0$指的是只含有零向量的线性子空间,零空间是核空间$\mathrm{Ker}\,\bm{\varphi}$.
    显然\[
        0\subseteq \mathrm{Ker}\,\bm{\varphi}
    \]
}
\exa{}{}{
    考虑$V=\mathbb{R}^2$,线性变换$\bm{\varphi}$为绕原点逆时针旋转$\theta$角,
    其在标准基下的表示矩阵为
    \[
        \bm{A}=\begin{bmatrix}
            \cos \theta & -\sin \theta \\
            \sin \theta & \cos \theta
        \end{bmatrix}
    \]
    设$L$为一维$\bm{\varphi}$-不变子空间即$\bm{\varphi}\left(
        L
        \right)\subseteq L$,容易知道
    $\theta \neq k\pi\left(k\in \mathbb{Z}\right)$
    时这样的$L$不存在即此时$\bm{\varphi}$没有
    非平凡的不变子空间.
}
\exa{}{}{
    考虑$\bm{\varphi}:\bm{\alpha}\longmapsto k\bm{\alpha}
        \left(k\in \mathbb{K}\text{固定}\right)$,则
    $V$的任一子空间均为$\bm{\varphi}$-不变子空间.
}
\lem{}{张成的子空间为不变子空间}{
    设线性变换$\bm{\varphi}\in \mathcal{L}\left(V
        \right)$,子空间$U=L\left(\bm{\alpha}_1,\bm{\alpha}_2,\cdots
        ,\bm{\alpha}_m\right)$,其中$\bm{\alpha}_i\in V$,则
    $U$为$\bm{\varphi}$-不变子空间等价于
    \[
        \bm{\varphi}\left(\bm{\alpha}_i\right)
        \in U\left(\forall 1 \leqslant i \leqslant m \right)
    \]\begin{proof}
        必要性显然. 下证充分性.

        设$\bm{\varphi}\left(\bm{\alpha}_i\right)\in U\left(1\leqslant i\leqslant m\right)$,任取$\bm{\alpha}
            \in U$. 故有
        \[
            \bm{\alpha}=\lambda_1\bm{\alpha}_1+\lambda_2\bm{\alpha}_2+\cdots+\lambda_m\bm{\alpha}_m
        \]
        则两边作用$\bm{\varphi}$即证.
    \end{proof}
}
\subsection{表示矩阵}
\thm{}{不变子空间诱导的线性变换的表示矩阵}{
    设线性变换$\bm{\varphi}\in \mathcal{L}\left(V^n\right)$,
    且$U$为$\bm{\varphi}$-不变子空间,取$U$的一组基
    为$\left\{\bm{e}_1,\bm{e}_2,
        \cdots,\bm{e}_r\right\}$,将其扩张为
    $V$的一组基$\left\{\bm{e}_1,\bm{e}_2,\cdots,\bm{e}_r,\bm{e}_{r+1},\cdots,
        \bm{e}_n\right\}$,则$\bm{\varphi}$在这一组基下的表示矩阵的形式为
    \[
        \begin{pmatrix}
            \bm{A}^r & \bm{B}\quad\, \\
            \bm{O}   & \bm{D}^{n-r}
        \end{pmatrix}
    \]\begin{proof}
        因为$\bm{\varphi}\left(\bm{e}_i\right)\in U\left(1\leqslant i\leqslant r\right)$,即有
        \[
            \begin{cases*}
                \bm{\varphi}\left(\bm{e}_1\right)=a_{11}\bm{e}_1+a_{21}\bm{e}_2+\cdots+a_{r1}\bm{e}_r \\
                \bm{\varphi}\left(\bm{e}_2\right)=a_{12}\bm{e}_1+a_{22}\bm{e}_2+\cdots+a_{r2}\bm{e}_r \\
                \qquad\qquad\cdots\cdots\cdots\cdots\cdots                                            \\
                \bm{\varphi}\left(\bm{e}_r\right)=a_{1r}\bm{e}_1+a_{2r}\bm{e}_2+\cdots+a_{rr}\bm{e}_r
            \end{cases*}
        \]
        于是
        \begin{align*}
              & \left(\bm{\varphi}\left(\bm{e}_1\right),
            \bm{\varphi}\left(\bm{e}_2\right),
            \cdots,\bm{\varphi}\left(\bm{e}_r\right),
            \bm{\varphi}\left(\bm{e}_{r+1}\right),
            \cdots,\bm{\varphi}\left(\bm{e}_n\right)\right)
            \\
            = &
            \left(\bm{e}_1,\bm{e}_2,\cdots,\bm{e}_r,\bm{e}_{r+1},\cdots,\bm{e}_n\right)
            \begin{bmatrix}
                \bm{A}^r & \bm{B}\quad\, \\
                \bm{O}   & \bm{D}^{n-r}
            \end{bmatrix}
            \qedhere
        \end{align*}
    \end{proof}
}
\cref{thm:不变子空间诱导的线性变换的表示矩阵}的逆命题也是成立的,即\thm{}{线性变换的限制诱导的不变子空间}{
    设线性变换$\bm{\varphi}\in \mathcal{L}\left(
        V^n\right)$,且$\bm{\varphi}$在$V$的一组基
    $\left\{\bm{e}_1,\bm{e}_2,\cdots,
        \bm{e}_n\right\}$下的表示矩阵具有形式
    \[
        \begin{pmatrix}
            \bm{A}^r & \bm{B}\quad\, \\
            \bm{O}   & \bm{D}^{n-r}
        \end{pmatrix}
    \]
    那么$U=L\left(\bm{e}_1,\bm{e}_2,\bm{e}_r\right)$
    是$\bm{\varphi}$-不变子空间.
}
\cor{}{不变子空间分解全空间}{
    设线性变换$\bm{\varphi}\in \mathcal{L}\left(V^n\right)$,
    且$V=V_1\oplus V_2$($V_1$、$V_2$均为$\bm{\varphi}$-不变子空间),
    那么一定可以分别取$V_1$、$V_2$的一组基来拼成$V$的一组基,使得
    $\bm{\varphi}$在这组基下的表示矩阵为
    \[
        \begin{pmatrix}
            \bm{A}^r & \bm{O}\quad\, \\
            \bm{O}   & \bm{D}^{n-r}
        \end{pmatrix}
    \]\begin{proof}
        取$V_1$的基$\left\{\bm{e}_1,\bm{e}_2,\cdots,\bm{e}_r\right\}$,
        $V_2$的基$\left\{\bm{e}_{r+1},\cdots,\bm{e}_n\right\}$. 因为
        $\bm{\varphi}\left(\bm{e}_i\right)\in V_1\left(\forall 1\leqslant i\leqslant r\right)$,
        $\bm{\varphi}\left(\bm{e}_i\right)\in V_2\left(\forall r+1\leqslant i\leqslant n\right)$.
        仿照\cref{thm:不变子空间诱导的线性变换的表示矩阵}即得.
    \end{proof}
}
\cor{}{全空间分解为不变子空间}{
设线性变换$\bm{\varphi}\in \mathcal{L}\left(
    V^n
    \right)$,且$V=V_1\oplus V_2\oplus \cdots \oplus V_m$($V_i$均为
$\bm{\varphi}$-不变子空间),分别取$V_i$的基,设$\bm{\varphi}\big |_{V_i}$在
给定基下的表示矩阵为$\bm{A}_i\left(1\leqslant i \leqslant m\right)$,
则将这些$V_i$的基拼成$V$的一组基后,
$\bm{\varphi}$在这组基下的表示矩阵为\[\bm{A}=
    \mathrm{diag}\left\{
    \bm{A}_1,\bm{A}_2,\cdots,\bm{A}_m
    \right\}\]
}
\subsection{例子}
\exa{}{}{
    设$V^3$的一组基为$\left\{\bm{e}_1,\bm{e}_2,\bm{e}_3\right\}$,
    线性变换$\bm{\varphi}\in \mathcal{L}\left(V\right)$在这组基下的表示矩阵为
    \[\bm{A}=
        \begin{pmatrix}
            3 & 1 & -1 \\
            2 & 2 & -1 \\
            2 & 2 & 0
        \end{pmatrix}
    \]
    求证:$U=L\left(\bm{e}_3,\bm{e}_1+\bm{e}_2+2\bm{e}_3\right)$为$\bm{\varphi}$-不变子空间.
    \begin{proof}
        只需证
        \[
            \bm{\varphi}\left(\bm{e}_3\right)\in U,
            \bm{\varphi}\left(\bm{e}_1+\bm{e}_2+2\bm{e}_3\right)\in U
        \]
        通过线性同构
        \[
            \bm{\eta}:V\longrightarrow \mathbb{K}^3
        \]
        我们仅需要验证坐标向量即可
        \[
            \bm{e}_3 \longmapsto \begin{pmatrix}
                0 \\
                0 \\
                1
            \end{pmatrix},\bm{e}_1+\bm{e}_2+2\bm{e}_3\longmapsto
            \begin{pmatrix}
                1 \\1\\2
            \end{pmatrix}
        \]
        作线性变换
        \[
            \bm{\varphi}\left(\bm{e}_3\right)=\bm{A}\begin{pmatrix}
                0 \\0\\1
            \end{pmatrix}=\begin{pmatrix}
                -1 \\-1\\0
            \end{pmatrix},\bm{\varphi}\left(\bm{e}_1+\bm{e}_2+2\bm{e}_3\right)=
            \bm{A}\begin{pmatrix}
                1 \\1\\2
            \end{pmatrix}=\begin{pmatrix}
                2 \\2\\4
            \end{pmatrix}
        \]
        只需要证明$\bm{\varphi}\left(\bm{e}_3\right)$和
        $\bm{\varphi}\left(\bm{e}_1+\bm{e}_2+2
            \bm{e}_3\right)$是$\bm{e}_3$、$\bm{e}_1
            +\bm{e}_2+2\bm{e}_3$的线性组合即可. 一般地,这就是解线性方程组
        \[
            \begin{pmatrix}
                0 & 1 \\0&1\\1&2
            \end{pmatrix}\bm{x}=\bm{\varphi}\left(\bm{e}_3\right)
            ,
            \begin{pmatrix}
                0 & 1 \\0&1\\1&2
            \end{pmatrix}\bm{x}=\bm{\varphi}\left(\bm{e}_1+\bm{e}_2+2\bm{e}_3\right)
        \]
        进一步地,只需要证明线性方程组
        有解即系数矩阵与增广矩阵的秩相等即可.
        \[
            \mathrm{rank}\begin{pmatrix}
                0 & 1 \\0&1\\1&2
            \end{pmatrix}=\mathrm{rank}\begin{pmatrix}
                0 & 1 & -1 \\0&1&-1\\1&2&0
            \end{pmatrix}=2
        \]
        \[
            \mathrm{rank}\begin{pmatrix}
                0 & 1 \\0&1\\1&2
            \end{pmatrix}=\mathrm{rank}\begin{pmatrix}
                0 & 1 & 2 \\0&1&2\\1&2&4
            \end{pmatrix}=2
            \qedhere
        \]
    \end{proof}
}